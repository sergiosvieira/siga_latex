\section{Referência do Arquivo G\+W\+L\+F\+Conf\+File.\+cpp}
\label{_g_w_l_f_conf_file_8cpp}\index{G\+W\+L\+F\+Conf\+File.\+cpp@{G\+W\+L\+F\+Conf\+File.\+cpp}}
{\ttfamily \#include \char`\"{}Hydrology/\+G\+W\+L\+F/\+G\+W\+L\+F\+Conf\+File.\+h\char`\"{}}\\*
{\ttfamily \#include \char`\"{}Hydrology/\+G\+W\+L\+F/\+G\+W\+L\+F\+Model.\+h\char`\"{}}\\*
{\ttfamily \#include \char`\"{}Hydrology/\+G\+W\+L\+F/\+G\+W\+L\+F\+Scenario.\+h\char`\"{}}\\*
{\ttfamily \#include \char`\"{}Hydrology/\+G\+W\+L\+F/\+G\+W\+L\+F\+Snow\+Pack.\+h\char`\"{}}\\*
{\ttfamily \#include \char`\"{}Hydrology/\+G\+W\+L\+F/\+G\+W\+L\+F\+Runoff.\+h\char`\"{}}\\*
{\ttfamily \#include \char`\"{}Hydrology/\+G\+W\+L\+F/\+G\+W\+L\+F\+Soil\+Water.\+h\char`\"{}}\\*
{\ttfamily \#include \char`\"{}Hydrology/\+G\+W\+L\+F/\+G\+W\+L\+F\+Model\+Main.\+h\char`\"{}}\\*
{\ttfamily \#include \char`\"{}Hydrology/\+G\+W\+L\+F/\+G\+W\+L\+F\+Interface.\+h\char`\"{}}\\*
{\ttfamily \#include \char`\"{}Hydrology/\+Hidro\+Main.\+h\char`\"{}}\\*
{\ttfamily \#include \char`\"{}Network/\+Subbasin.\+h\char`\"{}}\\*
{\ttfamily \#include \char`\"{}S\+I\+G\+A/\+Serie\+File.\+h\char`\"{}}\\*
